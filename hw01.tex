   \documentclass [11pt]{article}
   %\usepackage{ulem}
   \usepackage{latexsym}
   \usepackage{amssymb}
   \usepackage{amsmath}
   \usepackage{amsthm}
   \usepackage{mathrsfs} %Support use of the Raph Smith’s Formal Script font in mathematics
   \usepackage[sc]{mathpazo} %mathematical fonts
   \usepackage{url}
   \usepackage{ulem}
   \usepackage{enumitem}

   \textwidth      15cm
   \textheight     23cm
   \oddsidemargin 0.5cm
   \topmargin    -0.5cm
   \evensidemargin\oddsidemargin


   \pagestyle{plain}


\title{Complexity Theory}
\author{VU 181.142, SS 2014}
\date{Homework Assignment 1}
 
  \newtheorem{theorem}{Theorem}
  \newtheorem{lemma}[theorem]{Lemma}
  \newtheorem{exercise}{Exercise}

  \newcommand{\ra}{\rightarrow}
  \newcommand{\Ra}{\Rightarrow}
  \newcommand{\La}{\Leftarrow}
  \newcommand{\la}{\leftarrow}
  \newcommand{\LR}{\Leftrightarrow}


\newcommand{\compl}[1]{\overline{#1}}

\renewcommand{\phi}{\varphi}
\renewcommand{\theta}{\vartheta}

\def\II{{\cal I}} 
\newcommand{\True}{\mathbf{true}}
\newcommand{\False}{\mathbf{false}}
\def\ox{\neg x} 
\def\oy{\neg y} 
\def\oz{\neg z} 
\def\oa{\overline{\alpha}}
\def\ob{\overline{\beta}}
\def\og{\overline{\gamma}}
\def\od{\overline{\delta}}
\def\oe{\overline{\epsilon}}
\def\oz{\overline{\zeta}}
\def\a{\alpha}
\def\b{\beta}
\def\g{\gamma}
\def\d{\delta}
\def\e{\epsilon}
\def\z{\zeta}

\newcommand\numberthis{\addtocounter{equation}{1}\tag{\theequation}}
% within an align* environment, number and label the equation. #1 is the 
%label text to be appended to eqn:
\newcommand\neqn[1]{\numberthis\label{eqn:#1}}

\renewcommand{\em}{\rm}

%%%%%%%%%%%%%%%%%%%%%%%%%%%%%%%%%%%%%%%%%%%%%%%%%%%%%%%%%%%%%%%%%%%

\begin{document}


\maketitle

\begin{tabbing}
Submission Deadline: \quad \= \kill
Name: \> Martin Kalany \\
Matr-Nr: \> 0825673 \\
Begin: \> 18 March, 2014  \\
Submission Deadline: \>1 April, 2014  \\
send to: \> \url{complexity@dbai.tuwien.ac.at}\\
Maximum credits: \> 10 
\end{tabbing}


\medskip


\noindent
In the lecture ``Formale Methoden der Informatik'', the following Cook reduction from {\bf co-2-SAT}
to the {\bf REACHABILITY} problem was given:

\begin{itemize}
\item The variables of $\phi$ and their negations form the vertices of
$G(\phi)$.
\item
There is an arc $(\alpha,\beta)$ iff there is a clause $\compl{\alpha}
\lor \beta$ or $\beta \lor \compl{\alpha}$ in $\phi$, 
where $\compl{\alpha}$ is the complement of $\alpha$, i.e.: If $\alpha$ is true
in some satisfying assignment ${\cal I}$ of $\phi$, then $\beta$ must also be true in ${\cal I}$.
\item It can be shown that $\phi$ is unsatisfiable iff there is a variable $x$ such that
there are paths from $x$ to $\neg x$ and from $\neg x$ to $x$ in
$G(\phi)$.
\end{itemize}

\noindent
{\em Notation.}
It is convenient to write $x \Ra y$ if $y$ is reachable from $x$ in 
the graph $G(\phi)$.

\bigskip

\noindent
{\em \bf{Remark.}}
I find it a bit confusing that $x$ and $z$ are used both for literals and variables. Thus, let $x$, $y$, $z$ be variables and let $\a$, $\b$, $\g$, $\d$, $\e$, $\z$ be literals. I modified the given algorithm accordingly. 

\smallskip
\noindent
{\em \bf{Remark.}}
As usual, for any literals $\a$ and $\b$, we will assume that the commutative law holds, that is:
\begin{align}
(\a \lor \b) \iff (\b \lor \a) \quad .
\end{align}

\medskip

\noindent
\textbf{Definition.} Let $Path(\a,\b)$ denote a sequence of arcs of the form 
\begin{align}
Path(\a,\b) := (\a, \g_1),(\g_1,\g_2),\dots(\g_{i-i},\g_i),(\g_i,\b) \quad ,
\end{align}
where $(\a,\g_1) \in G(\phi)$, $(\g_i,\b) \in G(\phi)$ and $\forall\; 2 \leq j \leq i:\;(\g_{j-1}, \g_j) \in G(\phi)$. W.l.o.g., assume that this path is cycle free, i.e., that $\a\neq \g_1$, $\g_i \neq\b$ and $\g_j \neq \g_k\;\forall \leq j, k \leq i$, $j\neq k$.

\medskip

\begin{exercise}[4 credits]\label{ex1}
{\em Give a rigorous proof of the 
``if''-direction of the correctness of the above reduction, i.e.: \\
%
If there exists a variable~$x$, s.t.\
$x \Ra  \neg x$ and $\neg x \Ra x$ in
$G(\phi)$, then $\phi$ is unsatisfiable.
}%em
\end{exercise}


\noindent
{\bf Hint.} Carefully distinguish between what is assumed, what is defined and what has to be shown. The proof could thus start as follows: 

\medskip
\noindent
{\bf Proof.} (indirect) 
Suppose that there exists a variable~$x$, s.t.\
$x \Ra  \neg x$ and $\neg x \Ra x$ in
$G(\phi)$. Moreover, suppose that there exists a model ${\cal I}$ of $\phi$. We derive a contradiction by showing that 
then both $x$ and $\neg x$ are true in ${\cal I}$.

\bigskip
\noindent
Suppose 
\begin{align}
\exists x \in G(\phi):\;x \Ra  \ox \quad ,
\end{align}
i.e., $\exists\; Path(x, \ox) \in G(\phi)$.
By construction of $G(\phi)$ (given by the reduction), an arc $(\alpha, \beta) \in G(\phi) \Leftrightarrow (\compl{\alpha} \lor \beta)$ in $\phi$; thus, $(\compl{x} \lor \g_1)$, $(\compl{\g_i} \lor \compl{x})$, $\forall 2 \leq j \leq i:\;(\compl{\g_{j-1}} \lor \g_j)$  in $\phi$. Since $\phi$ is a formula in 2-CNF, we can equally write 
\begin{align}
\phi' := (\compl{x} \lor \g_1) \land (\compl{\g_1} \lor \g_2) \land \dots \land (\compl{\g_{i-1}}, \g_i) \land (\compl{\g_i} \lor \compl{x}) \quad ,
\end{align}
with $\phi'$ in $\phi$. For a 2-CNF formula to be satisfied under assignment ${\cal I}$, each clause must evaluate to true (under assignment ${\cal I}$). We proceed by case distinction on the truth value assigned to $\g_1$ by ${\cal I}$:
\begin{itemize}
\item ${\cal I}(\g_1)$ = false: The clause $(\compl{x} \lor \g_1)$ in $\phi$ immediately implies that ${\cal I}(x)$ = false, otherwise ${\cal I}(\phi)$ cannot evaluate to true.
\item ${\cal I}(\g_1)$ = true: We make the following observation:
\begin{align}
(\ox \lor \g_1) \land (\compl{\g_1} \lor \g_2) \land {\cal I}(\g_1) = true \implies {\cal I}(\g_2) = true \quad . \neqn{observation}
\end{align}
In general, 
\begin{align}
(\oa \lor \b) \land (\ob \lor \g)\;\;,\;\;{\cal I}(\b) = true \implies {\cal I}(\g) = true \quad . \neqn{observation} 
\end{align}
Applying this argument for all $i$, we get that ${\cal I}(\g_i)$ = true, which, in conjunction with the clause $(\compl{\g_i} \lor \ox)$, implies that ${\cal I}(x)$ = false.
\end{itemize}
Thus, 
\begin{align}
\exists x \in G(\phi):\;x \Ra  \ox \implies {\cal I}(x) = false \quad . \neqn{res1}
\end{align}

\bigskip
\noindent
Next, suppose 
\begin{align}
\exists x \in G(\phi):\;\ox \Ra  x \quad .
\end{align}
By the same argument as above, we get
\begin{align}
(x \lor \g_1) \land (\compl{\g_1} \lor \g_2) \land \dots \land (\compl{\g_{i-1}}, \g_i) \land (\compl{\g_i} \lor x) = \phi''
\end{align}
with $\phi''$ in $\phi$. By case distinction on $\II(\g_1)$:
\begin{itemize}
\item ${\cal I}(\g_1)$ = false: The clause $(x \lor \g_1)$ in $\phi$ immediately implies that ${\cal I}(x)$ = true, otherwise ${\cal I}(\phi)$ cannot evaluate to true.
\item ${\cal I}(\g_1)$ = true: Using Eq. \ref{eqn:observation}, we get that ${\cal I}(\g_i)$ = true, which, in conjunction with the clause $(\compl{\g_i} \lor x)$, implies that ${\cal I}(x)$ = true.
\end{itemize}
Thus, 
\begin{align}
\exists x \in G(\phi):\;\ox \Ra  x \implies {\cal I}(x) = true\quad . \neqn{res2}
\end{align}

Eq. \ref{eqn:res1} and Eq. \ref{eqn:res2} form a contradiction, since $x$ cannot be both true and false under assignment  ${\cal I}$. The claim from Exercise \ref{ex1} follows directly.
\qed

\bigskip


\noindent
For the ``only if''-direction of the correctness proof of the 
problem reduction from {\bf co-2-SAT}
to {\bf REACHABILITY}, we have to show the following implication: If there exists no 
variable~$x$, s.t.\
$x \Ra  \ox$ and $\ox \Ra x$ in
$G(\phi)$, then there 
exists a model $\II$ of $\phi$.
To this end, consider the truth assignment $\II$ constructed by the following algorithm: 
%
%
\begin{tabbing}
123\=123\=123\=123\=123\=123\=\kill
/* Step 1 */  \\
{\bf for each} literal $\alpha$ \sout{$x$}, s.t.\ $\overline{\alpha} \Ra \alpha$ \sout{$\overline{x} \Ra x$} 
{\bf do} \\
\> $\II(\alpha) := \True$; \sout{$\II(x) := \True$;}
 /* hence, implicitly, $\II(\oa) := \False$; \sout{$\II(\ox) := \False$;} */ \\
\> {\bf for each} literal $\beta$ \sout{$z$} with $\alpha \Ra \beta$ \sout{$x \Ra z$} 
{\bf do}  $\II(\beta) := \True$ \sout{$\II(z) := \True$} {\bf od};
\\
{\bf od}; 
\\[1.1ex]
/* Step 2 */  \\
{\bf while} there exists $x \in Var(\phi)$, s.t.\ $\II(x)$ is undefined
{\bf do} \\
\> choose an arbitrary variable $x$, s.t.\ $\II(x)$ is undefined; \\
\> $\II(x) := \True$;  \\
\> {\bf for each} literal $\beta$ \sout{$z$} with $x \Ra \beta$ \sout{$x \Ra z$}
{\bf do}  $\II(\beta) := \True$ \sout{$\II(z) := \True$} {\bf od};\\
{\bf od};
\end{tabbing}
%
%
To show that an assignment $\II$ thus constructed is indeed a 
model of $\phi$, it is convenient to proof the following lemmas.

\begin{lemma}
\label{lem:1}
Suppose that there exists no
variable $x$, s.t.\ both $x \Ra \ox$ and $\ox \Ra x$ hold.
Then a truth assignment made by the above algorithm is never changed later, i.e.: it cannot happen, that at some stage, 
$\II (z) =\True$ for some variable $z$ and later this value is changed to 
$\II (z) =\False$ or vice versa.
\end{lemma}

\begin{lemma}
\label{lem:2}
Suppose that there exists no
variable $x$, s.t.\ both $x \Ra \ox$ and $\ox \Ra x$ hold.
Then the truth assignment $\II$ constructed by the above algorithm
is a model of $\phi$.
\end{lemma}

\noindent
We first state and prove Lemma~\ref{lem:handy}, which will come in handy for the proofs of Lemma~\ref{lem:1} and Lemma~\ref{lem:2}:
\begin{lemma}
\label{lem:handy}
Let $\a$, $\b$ be arbitrary literals. Then, 
\begin{align}
\a \Ra \b \implies \ob \Ra \oa
\end{align}
\end{lemma}

\noindent
{\bf Proof.}
\begin{align}
\a \Ra \b \implies \exists\; Path(\a, \b) =(\a, \g_1),(\g_1,\g_2),\dots,(\g_{i-1},\g_i),(\g_i,\b)
\end{align}
By the reduction
\begin{align}
(\oa \lor \g_1)\land (\compl{\g_1} \lor \g_2) \land \dots \land (\compl{\g_{i-1}} \lor \g_i) \land (\compl{\g_i} \lor \b)\; \mathrm{in}\; \phi \quad ,
\end{align}
and thus (by the reduction) arcs $(\ob, \compl{\g_{i}})$, $\forall 1 \leq j \leq i\; (\compl{\g_j}, \compl{\g_{j-1}})$ and $(\compl{\g_1}, \oa) \in G(\phi)$, i.e., $\ob \Ra \oa$. 
\qed

\smallskip

\noindent
\begin{exercise}[4 credits]
{\em Give a rigorous proof of Lemma \ref{lem:1}.
}%em
\end{exercise}
\noindent
{\bf Proof.} For any variable $x \in Var(\phi)$, one of the following three cases holds (by the assumption in Lemma 1):
\begin{enumerate}
\item $\ox \Ra x \in G(\phi)\;\;\land\;\; x \Ra \ox \not \in G(\phi)$.
\item $x \Ra \ox \in G(\phi) \;\;\land\;\;  \ox \Ra x \not \in G(\phi)$.
\item $\ox \Ra x \not \in G(\phi)\;\;\land\;\; x \Ra \ox \not \in G(\phi)$.
\end{enumerate}
For each case we will show by contradiction that for any $x \in Var(\phi)$, the algorithm never sets $\II(x) = \True\;(\False)$ after setting $\II(x) = \False\;(\True)$.
\begin{enumerate}[label=(\roman*)]
\item Case 1: Let $\a = x$. Then, in step 1.1, $\II(x) = \True$ and 
\begin{enumerate}[label=\alph*)]
\item step 1.1 will never assign a value to $x$ again (since we just took care of $\ox \Ra x$ and  $x \Ra \ox \not \in G(\phi)$);
\item in step 1.2, 
\begin{align}
\II(x) = \False \iff \b = \ox \land \II(\b) = \True \land (\a \Ra \b)
\end{align}
which implies that $x \Ra \ox$ holds. This contradicts the assumption.
\end{enumerate}
\item Case 2:  Let $\oa = x$. Then, in step 1.1, $\II(x) = \False$ and 
\begin{enumerate}[label=\alph*)]
\item step 1.1 will never assign a value to $x$ again;
\item in step 1.2, 
\begin{align}
\II(x) = \True \iff \b = x \land \II(\b) = \True \land (\a \Ra \b)
\end{align}
which implies that $\ox \Ra x$ holds and thus contradicts the assumption.
\end{enumerate}
\end{enumerate}
From (i) and (ii) we conclude that for cases 1 and 2, Lemma 1 holds for step 1 of the algorithm. For cases 1 and 2:
\begin{enumerate}[label=(\roman*),resume]
\item $x \in Var(\phi)$ and $\II(x)$ undefined is not possible (because $\II(x)$ was set in step 1 of the algorithm). Thus, step 2 will be skipped for $x$ and cannot change the assignment.
\item For $\II(x)$ to change in 
\begin{itemize}
\item Case 1 (note that (i) set $\II(x) = \True$): 
\begin{align}
\II(x) = \False \iff &\b = \ox \land \II(\b) = \True \land \exists y\neq x\; \\
& \mathrm{s.t.}\; \II(y)\; \mathrm{undefined} \land (y \Ra \b)\quad . \neqn{reqCase1}
\end{align}
However, $y \Ra \ox$ implies $x \Ra \oy$ (Lemma~\ref{lem:handy}), which in turn implies that $\II(y)$ was set in step 1.2, which was assumed undefined in Eq.~\ref{eqn:reqCase1}.
\item Case 2 (note that (ii) set $\II(x) = \False$): 
\begin{align}
\II(x) = \True \iff &\b = x \land \II(\b) = \True \land \exists y\neq x\; \\
& \mathrm{s.t.}\; \II(y)\; \mathrm{undefined} \land (y \Ra \b)\quad . \neqn{reqCase2}
\end{align}
Again, $y \Ra x$ implies $\ox \Ra \oy$ (Lemma~\ref{lem:handy}), which in turn implies that $\II(y)$ was set in step 1, which was assumed undefined in Eq.~\ref{eqn:reqCase2}.
\end{itemize}
\end{enumerate}
From (iii) and (iv) we conclude that for cases 1 and 2, Lemma 1 holds for step 2 of the algorithm as well (and thus Lemma 1 holds for cases 1 and 2).
\begin{enumerate}[label=(\roman*), resume]
\item Case 3: In step 2 of the algorithm, for any $x \in Var(\phi)$ s.t. $\II(x)$ is undefined, either:
\begin{itemize}
\item step 2.1 sets $\II(x) = \True$ at some point. To change the assignment,
\begin{align}
\II(x) = \False \iff &\b = \ox \land \II(\b) = \True \land \exists y\neq x\; \\
& \mathrm{s.t.}\; \II(y)\; \mathrm{undefined} \land (y \Ra \b)\quad . \neqn{reqCase31}
\end{align}
which, by Lemma~\ref{lem:handy}, implies that $x \Ra \oy$, and thus $\II(y)$ was already set in step 2.2 right after  $\II(x) = \True$ was set, violating Eq.~\ref{eqn:reqCase31}. 
\item $\II(x) = \True$ set by step 2.2 at some point requires 
\begin{align}
\II(x) = \True \iff &\b = x \land \II(\b) = \True \land \exists y\neq x\; \\
& \mathrm{s.t.}\; \II(y)\; \mathrm{undefined} \land (y \Ra \b)\quad . \neqn{reqCase32}
\end{align}
which, by Lemma~\ref{lem:handy}, implies that $x \Ra y$. To change the assignment,
\begin{align}
\II(x) = \False \iff &\b = \ox \land \II(\b) = \True \land \exists y\neq x\; \\
& \mathrm{s.t.}\; \II(y)\; \mathrm{undefined} \land (y \Ra \b)\quad . \neqn{reqCase33}
\end{align}
implying $x \Ra \oy$, which contradicts Eq.~\ref{eqn:reqCase 33} (by the same argument used in the above item).
\item $\II(x) = \False$ set by step 2.2 at some point requires 
\begin{align}
\II(x) = \False \iff &\b = \ox \land \II(\b) = \True \land \exists y\neq x\; \\
& \mathrm{s.t.}\; \II(y)\; \mathrm{undefined} \land (y \Ra \b)\quad . \neqn{reqCase34}
\end{align}
implies that $\ox \Ra y$. To change the assignment,
\begin{align}
\II(x) = \True \iff &\b = x \land \II(\b) = \True \land \exists z\neq x,y\; \\
& \mathrm{s.t.}\; \II(z)\; \mathrm{undefined} \land (z \Ra \b)\quad . \neqn{reqCase35}
\end{align}
implying $\ox \Ra \oz$ (Lemma~\ref{lem:handy}), which contradicts Eq.~\ref{eqn:reqCase35} (by the same argument used in the above item).
\end{itemize}
\end{enumerate}
From (v) we conclude that Lemma 1 holds for case 3 as well and thus for all cases.
\qed


\noindent
\begin{exercise}[2 credits]
{\em Give a rigorous proof of Lemma \ref{lem:2}.
}%em
\end{exercise}
\noindent
{\bf Proof.} Due to Lemma~\ref{lem:2}, it suffices to show that:
\begin{enumerate}
\item All $x \in Var(\phi)$ are assigned either $\II(x) = \True$ or $\II(x) = \False$. \\
This is trivially true since, in step 2
\begin{align}
\exists x \in Var(\phi)\; \mathrm{s.t.}\; \II(x)\; \mathrm{undefined} \implies \II(x) = \True
\end{align}
\item No assignment made by the algorithm causes a clause to evaluate to false.
\begin{enumerate}[label=(\roman*)]
\item Let $X_1 := \{x\in Var(\phi)\; |\; \ox \Ra x \}$. The assignment in step 1.1 ($\a = x$, $\II(\a) = \True$) causes clauses $\forall\; x \in X_1\;(x \lor \gamma)$ to evaluate to true. 
\item Let $B_1 := \{\b\; |\; x \Ra \b \}$. The assignment in step 1.2 ($\forall\; \b \in B_1: \;\II(\b) = \True$) causes all clauses $\forall\; \b \in B_1\;(\b \lor \gamma)$ to evaluate to true. Note that this includes all clauses $(\d \lor \ox)$ for any $\d$ (we can set $\g = \ox$ and $\forall\;\d: \d \in B_1$ since $(\d \lor \ox)$ in $\phi$ implies $(x\Ra \d) \in G(\phi)$).
\item Let $X_2 := \{x\in Var(\phi)\; |\; x \Ra \ox \}$. The assignment in step 1.1 ($\a = \ox$, $\II(\a) = \True$ and thus $\II(x) = false$) causes clauses $\forall\; x \in X_2\;(\ox, \gamma)$ to evaluate to true.
\item Let $B_2 := \{\b\; |\; \ox \Ra \b \}$. The assignment in step 1.2 ($\forall\; \b \in B_1: \;\II(\b) = \True$) causes clauses $\forall\; \b \in B_1\;(\b \lor \gamma)$ to evaluate to true. Note that this includes all clauses $(\d \lor x)$ for any $\d$.
\end{enumerate}
Let $X := X_1 \cup X_2$ and $B := B_1 \cup B_2$.
\begin{enumerate}[label=(\roman*), resume]
\item From (i) to (iv) we can conclude that, after step 1, only clauses of the form $(\g \lor \d)$, where 
\begin{itemize}
\item $Var(\g) \not \in X$
\item $Var(\d) \not \in X$
\item $\g \not \in B$
\item $\d \not \in B$,
\end{itemize}
do not yet evaluate to true. Equivalently, $\II(\d)$ undefined and $\II(g)$ undefined (otherwise, one of the above properties would be violated). 
\begin{itemize}
\item W.l.o.g., let $d=Var(\d)$ be the variable chosen in Step 2. The assignment in step 2.1 causes clauses of the form $(d \lor \e)$ to evaluate to true. 
\item Let $B_3 := \{\b\; |\; d \Ra \b \}$. The assignment in step 2.2 ($\forall\; \b \in B_3:\;\II(\b) = \True$) causes all clauses $\forall\; \b \in B_1\;(\b \lor \e)$ to evaluate to true. Note that this includes the clauses $(\z \lor \neg d)$ for any literal $\z$ (we can set $\e = \neg d$ and $\forall\;\z: \z \in B_3$ since $(\z \lor \neg d)$ implies $(d \Ra \z)$).
\item The two items above imply that for any variable $d \in Var(\phi$) with $\II(d)$ undefined before step 2 of the algorithm, all clauses of the form $(d,\e)$ or $(\neg d, \e)$ evaluate to true after step 2. 
\end{itemize}
\end{enumerate}
\end{enumerate}

\noindent
This concludes the proof of Lemma~\ref{lem:2}. since we showed that 1.~all variables in $\phi$ are assigned a truth value and 2.~no assignment causes a clause to evaluate to false.

\qed


\end{document}


